% Options for packages loaded elsewhere
% Options for packages loaded elsewhere
\PassOptionsToPackage{unicode}{hyperref}
\PassOptionsToPackage{hyphens}{url}
\PassOptionsToPackage{dvipsnames,svgnames,x11names}{xcolor}
%
\documentclass[
  11pt,
  letterpaper,
  DIV=11,
  numbers=noendperiod]{scrartcl}
\usepackage{xcolor}
\usepackage[margin=1in]{geometry}
\usepackage{amsmath,amssymb}
\setcounter{secnumdepth}{-\maxdimen} % remove section numbering
\usepackage{iftex}
\ifPDFTeX
  \usepackage[T1]{fontenc}
  \usepackage[utf8]{inputenc}
  \usepackage{textcomp} % provide euro and other symbols
\else % if luatex or xetex
  \usepackage{unicode-math} % this also loads fontspec
  \defaultfontfeatures{Scale=MatchLowercase}
  \defaultfontfeatures[\rmfamily]{Ligatures=TeX,Scale=1}
\fi
\usepackage{lmodern}
\ifPDFTeX\else
  % xetex/luatex font selection
\fi
% Use upquote if available, for straight quotes in verbatim environments
\IfFileExists{upquote.sty}{\usepackage{upquote}}{}
\IfFileExists{microtype.sty}{% use microtype if available
  \usepackage[]{microtype}
  \UseMicrotypeSet[protrusion]{basicmath} % disable protrusion for tt fonts
}{}
\makeatletter
\@ifundefined{KOMAClassName}{% if non-KOMA class
  \IfFileExists{parskip.sty}{%
    \usepackage{parskip}
  }{% else
    \setlength{\parindent}{0pt}
    \setlength{\parskip}{6pt plus 2pt minus 1pt}}
}{% if KOMA class
  \KOMAoptions{parskip=half}}
\makeatother
% Make \paragraph and \subparagraph free-standing
\makeatletter
\ifx\paragraph\undefined\else
  \let\oldparagraph\paragraph
  \renewcommand{\paragraph}{
    \@ifstar
      \xxxParagraphStar
      \xxxParagraphNoStar
  }
  \newcommand{\xxxParagraphStar}[1]{\oldparagraph*{#1}\mbox{}}
  \newcommand{\xxxParagraphNoStar}[1]{\oldparagraph{#1}\mbox{}}
\fi
\ifx\subparagraph\undefined\else
  \let\oldsubparagraph\subparagraph
  \renewcommand{\subparagraph}{
    \@ifstar
      \xxxSubParagraphStar
      \xxxSubParagraphNoStar
  }
  \newcommand{\xxxSubParagraphStar}[1]{\oldsubparagraph*{#1}\mbox{}}
  \newcommand{\xxxSubParagraphNoStar}[1]{\oldsubparagraph{#1}\mbox{}}
\fi
\makeatother

\usepackage{color}
\usepackage{fancyvrb}
\newcommand{\VerbBar}{|}
\newcommand{\VERB}{\Verb[commandchars=\\\{\}]}
\DefineVerbatimEnvironment{Highlighting}{Verbatim}{commandchars=\\\{\}}
% Add ',fontsize=\small' for more characters per line
\usepackage{framed}
\definecolor{shadecolor}{RGB}{241,243,245}
\newenvironment{Shaded}{\begin{snugshade}}{\end{snugshade}}
\newcommand{\AlertTok}[1]{\textcolor[rgb]{0.68,0.00,0.00}{#1}}
\newcommand{\AnnotationTok}[1]{\textcolor[rgb]{0.37,0.37,0.37}{#1}}
\newcommand{\AttributeTok}[1]{\textcolor[rgb]{0.40,0.45,0.13}{#1}}
\newcommand{\BaseNTok}[1]{\textcolor[rgb]{0.68,0.00,0.00}{#1}}
\newcommand{\BuiltInTok}[1]{\textcolor[rgb]{0.00,0.23,0.31}{#1}}
\newcommand{\CharTok}[1]{\textcolor[rgb]{0.13,0.47,0.30}{#1}}
\newcommand{\CommentTok}[1]{\textcolor[rgb]{0.37,0.37,0.37}{#1}}
\newcommand{\CommentVarTok}[1]{\textcolor[rgb]{0.37,0.37,0.37}{\textit{#1}}}
\newcommand{\ConstantTok}[1]{\textcolor[rgb]{0.56,0.35,0.01}{#1}}
\newcommand{\ControlFlowTok}[1]{\textcolor[rgb]{0.00,0.23,0.31}{\textbf{#1}}}
\newcommand{\DataTypeTok}[1]{\textcolor[rgb]{0.68,0.00,0.00}{#1}}
\newcommand{\DecValTok}[1]{\textcolor[rgb]{0.68,0.00,0.00}{#1}}
\newcommand{\DocumentationTok}[1]{\textcolor[rgb]{0.37,0.37,0.37}{\textit{#1}}}
\newcommand{\ErrorTok}[1]{\textcolor[rgb]{0.68,0.00,0.00}{#1}}
\newcommand{\ExtensionTok}[1]{\textcolor[rgb]{0.00,0.23,0.31}{#1}}
\newcommand{\FloatTok}[1]{\textcolor[rgb]{0.68,0.00,0.00}{#1}}
\newcommand{\FunctionTok}[1]{\textcolor[rgb]{0.28,0.35,0.67}{#1}}
\newcommand{\ImportTok}[1]{\textcolor[rgb]{0.00,0.46,0.62}{#1}}
\newcommand{\InformationTok}[1]{\textcolor[rgb]{0.37,0.37,0.37}{#1}}
\newcommand{\KeywordTok}[1]{\textcolor[rgb]{0.00,0.23,0.31}{\textbf{#1}}}
\newcommand{\NormalTok}[1]{\textcolor[rgb]{0.00,0.23,0.31}{#1}}
\newcommand{\OperatorTok}[1]{\textcolor[rgb]{0.37,0.37,0.37}{#1}}
\newcommand{\OtherTok}[1]{\textcolor[rgb]{0.00,0.23,0.31}{#1}}
\newcommand{\PreprocessorTok}[1]{\textcolor[rgb]{0.68,0.00,0.00}{#1}}
\newcommand{\RegionMarkerTok}[1]{\textcolor[rgb]{0.00,0.23,0.31}{#1}}
\newcommand{\SpecialCharTok}[1]{\textcolor[rgb]{0.37,0.37,0.37}{#1}}
\newcommand{\SpecialStringTok}[1]{\textcolor[rgb]{0.13,0.47,0.30}{#1}}
\newcommand{\StringTok}[1]{\textcolor[rgb]{0.13,0.47,0.30}{#1}}
\newcommand{\VariableTok}[1]{\textcolor[rgb]{0.07,0.07,0.07}{#1}}
\newcommand{\VerbatimStringTok}[1]{\textcolor[rgb]{0.13,0.47,0.30}{#1}}
\newcommand{\WarningTok}[1]{\textcolor[rgb]{0.37,0.37,0.37}{\textit{#1}}}

\usepackage{longtable,booktabs,array}
\usepackage{calc} % for calculating minipage widths
% Correct order of tables after \paragraph or \subparagraph
\usepackage{etoolbox}
\makeatletter
\patchcmd\longtable{\par}{\if@noskipsec\mbox{}\fi\par}{}{}
\makeatother
% Allow footnotes in longtable head/foot
\IfFileExists{footnotehyper.sty}{\usepackage{footnotehyper}}{\usepackage{footnote}}
\makesavenoteenv{longtable}
\usepackage{graphicx}
\makeatletter
\newsavebox\pandoc@box
\newcommand*\pandocbounded[1]{% scales image to fit in text height/width
  \sbox\pandoc@box{#1}%
  \Gscale@div\@tempa{\textheight}{\dimexpr\ht\pandoc@box+\dp\pandoc@box\relax}%
  \Gscale@div\@tempb{\linewidth}{\wd\pandoc@box}%
  \ifdim\@tempb\p@<\@tempa\p@\let\@tempa\@tempb\fi% select the smaller of both
  \ifdim\@tempa\p@<\p@\scalebox{\@tempa}{\usebox\pandoc@box}%
  \else\usebox{\pandoc@box}%
  \fi%
}
% Set default figure placement to htbp
\def\fps@figure{htbp}
\makeatother





\setlength{\emergencystretch}{3em} % prevent overfull lines

\providecommand{\tightlist}{%
  \setlength{\itemsep}{0pt}\setlength{\parskip}{0pt}}



 


\KOMAoption{captions}{tableheading}
\makeatletter
\@ifpackageloaded{caption}{}{\usepackage{caption}}
\AtBeginDocument{%
\ifdefined\contentsname
  \renewcommand*\contentsname{Table of contents}
\else
  \newcommand\contentsname{Table of contents}
\fi
\ifdefined\listfigurename
  \renewcommand*\listfigurename{List of Figures}
\else
  \newcommand\listfigurename{List of Figures}
\fi
\ifdefined\listtablename
  \renewcommand*\listtablename{List of Tables}
\else
  \newcommand\listtablename{List of Tables}
\fi
\ifdefined\figurename
  \renewcommand*\figurename{Figure}
\else
  \newcommand\figurename{Figure}
\fi
\ifdefined\tablename
  \renewcommand*\tablename{Table}
\else
  \newcommand\tablename{Table}
\fi
}
\@ifpackageloaded{float}{}{\usepackage{float}}
\floatstyle{ruled}
\@ifundefined{c@chapter}{\newfloat{codelisting}{h}{lop}}{\newfloat{codelisting}{h}{lop}[chapter]}
\floatname{codelisting}{Listing}
\newcommand*\listoflistings{\listof{codelisting}{List of Listings}}
\makeatother
\makeatletter
\makeatother
\makeatletter
\@ifpackageloaded{caption}{}{\usepackage{caption}}
\@ifpackageloaded{subcaption}{}{\usepackage{subcaption}}
\makeatother
\usepackage{bookmark}
\IfFileExists{xurl.sty}{\usepackage{xurl}}{} % add URL line breaks if available
\urlstyle{same}
\hypersetup{
  pdftitle={Exam 1 Prep},
  colorlinks=true,
  linkcolor={blue},
  filecolor={Maroon},
  citecolor={Blue},
  urlcolor={Blue},
  pdfcreator={LaTeX via pandoc}}


\title{Exam 1 Prep}
\author{}
\date{}
\begin{document}
\maketitle


\paragraph{\texorpdfstring{Define \textbf{variation} in your own words
and explain why it is central to
statistics.}{Define variation in your own words and explain why it is central to statistics.}}\label{define-variation-in-your-own-words-and-explain-why-it-is-central-to-statistics.}

\paragraph{\texorpdfstring{In the context of a data frame, what do
\textbf{rows} represent? What do \textbf{columns}
represent?}{In the context of a data frame, what do rows represent? What do columns represent?}}\label{in-the-context-of-a-data-frame-what-do-rows-represent-what-do-columns-represent}

\paragraph{\texorpdfstring{Which of the following is a
\emph{quantitative}
variable?}{Which of the following is a quantitative variable?}}\label{which-of-the-following-is-a-quantitative-variable}

\begin{enumerate}
\def\labelenumi{\Alph{enumi})}
\tightlist
\item
  Height
\item
  Favorite color
\item
  Gender
\item
  Eye color
\end{enumerate}

\paragraph{\texorpdfstring{Why is \textbf{sampling} necessary in
statistics? Provide one
reason.}{Why is sampling necessary in statistics? Provide one reason.}}\label{why-is-sampling-necessary-in-statistics-provide-one-reason.}

\paragraph{\texorpdfstring{What does it mean for variables to be
\textbf{categorical} versus \textbf{numerical}? Give an example of each
from a
dataset.}{What does it mean for variables to be categorical versus numerical? Give an example of each from a dataset.}}\label{what-does-it-mean-for-variables-to-be-categorical-versus-numerical-give-an-example-of-each-from-a-dataset.}

\paragraph{\texorpdfstring{In modeling notation, what do the symbols
\texttt{\textasciitilde{}} and \texttt{data=} indicate? Offer a brief
explanation.}{In modeling notation, what do the symbols \textasciitilde{} and data= indicate? Offer a brief explanation.}}\label{in-modeling-notation-what-do-the-symbols-and-data-indicate-offer-a-brief-explanation.}

\paragraph{\texorpdfstring{The vector
\texttt{x\ \textless{}-\ c(2,1,3,3,2,3,1,2,1)} is
given.}{The vector x \textless- c(2,1,3,3,2,3,1,2,1) is given.}}\label{the-vector-x---c213323121-is-given.}

\begin{enumerate}
\def\labelenumi{\Alph{enumi})}
\tightlist
\item
  After sorting \texttt{x}, what pattern becomes visible?
\item
  What does the frequency table of \texttt{x} show?
\end{enumerate}

\paragraph{\texorpdfstring{Using the \texttt{Fingers} dataset (from
class):}{Using the Fingers dataset (from class):}}\label{using-the-fingers-dataset-from-class}

\begin{enumerate}
\def\labelenumi{\Alph{enumi})}
\tightlist
\item
  What do boxplots of \texttt{Index\ \textasciitilde{}\ Gender} visually
  display about variability?
\item
  Describe center, spread, overlap, and any unusual features.
\end{enumerate}

\paragraph{Consider the following R
code:}\label{consider-the-following-r-code}

\begin{Shaded}
\begin{Highlighting}[]
\FunctionTok{gf\_histogram}\NormalTok{(}\SpecialCharTok{\textasciitilde{}}\NormalTok{Score, }\AttributeTok{data =}\NormalTok{ Data, }\AttributeTok{binwidth =} \DecValTok{2}\NormalTok{)}
\end{Highlighting}
\end{Shaded}

\begin{enumerate}
\def\labelenumi{\Alph{enumi})}
\item
  Interpret what the following R code does and what you would expect the
  plot to show:
\item
  What distribution characteristics would the histogram reveal?
\end{enumerate}

\paragraph{\texorpdfstring{Explain what the \textbf{five-number summary}
tells you about a numerical variable and relate it to
variation.}{Explain what the five-number summary tells you about a numerical variable and relate it to variation.}}\label{explain-what-the-five-number-summary-tells-you-about-a-numerical-variable-and-relate-it-to-variation.}

\paragraph{Write one sentence comparing the distributions displayed
by:}\label{write-one-sentence-comparing-the-distributions-displayed-by}

\begin{Shaded}
\begin{Highlighting}[]
\FunctionTok{gf\_histogram}\NormalTok{(}\SpecialCharTok{\textasciitilde{}}\NormalTok{Index, }\AttributeTok{data =}\NormalTok{ Fingers, }\AttributeTok{binwidth =} \FloatTok{0.25}\NormalTok{)}
\FunctionTok{gf\_histogram}\NormalTok{(}\SpecialCharTok{\textasciitilde{}}\NormalTok{Index, }\AttributeTok{data =}\NormalTok{ Fingers, }\AttributeTok{binwidth =} \FloatTok{0.5}\NormalTok{)}
\end{Highlighting}
\end{Shaded}

Focus on how the choice of \textbf{binwidth} affects the appearance.

\paragraph{\texorpdfstring{Explain in plain language what an
\textbf{outlier} is and how the 1.5×IQR rule identifies
outliers.}{Explain in plain language what an outlier is and how the 1.5×IQR rule identifies outliers.}}\label{explain-in-plain-language-what-an-outlier-is-and-how-the-1.5iqr-rule-identifies-outliers.}

\paragraph{\texorpdfstring{What does it tell you if two boxplots (for
\texttt{Index} by two groups) show a large difference in medians but
overlapping boxes? What does that say about variation within and between
groups?}{What does it tell you if two boxplots (for Index by two groups) show a large difference in medians but overlapping boxes? What does that say about variation within and between groups?}}\label{what-does-it-tell-you-if-two-boxplots-for-index-by-two-groups-show-a-large-difference-in-medians-but-overlapping-boxes-what-does-that-say-about-variation-within-and-between-groups}

\paragraph{consider the following R
output:}\label{consider-the-following-r-output}

\begin{Shaded}
\begin{Highlighting}[]
\FunctionTok{favstats}\NormalTok{(Fingers}\SpecialCharTok{$}\NormalTok{Pinkie)}
\end{Highlighting}
\end{Shaded}

\begin{verbatim}
 min Q1 median Q3 max     mean       sd   n missing
  33 55     58 63  98 59.41252 9.080594 157       0
\end{verbatim}

\begin{enumerate}
\def\labelenumi{\Alph{enumi})}
\item
  What can you say about the \emph{center} and \emph{spread} of the
  \texttt{Score} distribution?
\item
  Use 1.5IQR rule to find if there are any outliers.
\end{enumerate}

\paragraph{\texorpdfstring{Explain why it is important to identify the
\textbf{response variable} before choosing a plot to visualize a
relationship.}{Explain why it is important to identify the response variable before choosing a plot to visualize a relationship.}}\label{explain-why-it-is-important-to-identify-the-response-variable-before-choosing-a-plot-to-visualize-a-relationship.}

\paragraph{Given these R commands:}\label{given-these-r-commands}

\begin{Shaded}
\begin{Highlighting}[]
\FunctionTok{mean}\NormalTok{(Pinkie }\SpecialCharTok{\textasciitilde{}}\NormalTok{ Gender, }\AttributeTok{data =}\NormalTok{ Fingers) }\CommentTok{\# Returns means for Pinkie in different genders}
\FunctionTok{gf\_boxplot}\NormalTok{(Pinkie }\SpecialCharTok{\textasciitilde{}}\NormalTok{ Gender, }\AttributeTok{data =}\NormalTok{ Fingers)}
\end{Highlighting}
\end{Shaded}

Explain what each line does and how the two results complement each
other.

\paragraph{\texorpdfstring{Describe in words what the \textbf{residual}
represents in the context of a simple model predicting \texttt{Pinkie}
using \texttt{Gender}, and why smaller residuals imply a better
model.}{Describe in words what the residual represents in the context of a simple model predicting Pinkie using Gender, and why smaller residuals imply a better model.}}\label{describe-in-words-what-the-residual-represents-in-the-context-of-a-simple-model-predicting-pinkie-using-gender-and-why-smaller-residuals-imply-a-better-model.}

\paragraph{The following R code calculates
proportions:}\label{the-following-r-code-calculates-proportions}

\begin{Shaded}
\begin{Highlighting}[]
\FunctionTok{tally}\NormalTok{(Gender}\SpecialCharTok{\textasciitilde{}}\NormalTok{Job, }\AttributeTok{data=}\NormalTok{Fingers)}
\end{Highlighting}
\end{Shaded}

\begin{verbatim}
        Job
Gender   Not Working Part-time Job Full-time Job
  female          65            47             0
  male            25            19             1
\end{verbatim}

Give an example of a conditional probability and compute it.

\paragraph{\texorpdfstring{Below is R code that generates a conditional
proportion bar chart. Explain how this visualization helps you
\emph{explain variability in the
response}.}{Below is R code that generates a conditional proportion bar chart. Explain how this visualization helps you explain variability in the response.}}\label{below-is-r-code-that-generates-a-conditional-proportion-bar-chart.-explain-how-this-visualization-helps-you-explain-variability-in-the-response.}

\begin{Shaded}
\begin{Highlighting}[]
\FunctionTok{gf\_bar}\NormalTok{(}\SpecialCharTok{\textasciitilde{}}\NormalTok{Job, }\AttributeTok{data =}\NormalTok{ Fingers, }\AttributeTok{fill =} \SpecialCharTok{\textasciitilde{}}\NormalTok{Gender, }\AttributeTok{position =} \StringTok{"fill"}\NormalTok{)}
\end{Highlighting}
\end{Shaded}

\begin{Shaded}
\begin{Highlighting}[]
\FunctionTok{gf\_bar}\NormalTok{(}\SpecialCharTok{\textasciitilde{}}\NormalTok{Job, }\AttributeTok{data =}\NormalTok{ Fingers, }\AttributeTok{fill =} \SpecialCharTok{\textasciitilde{}}\NormalTok{Gender, }\AttributeTok{position =} \StringTok{"fill"}\NormalTok{)}
\end{Highlighting}
\end{Shaded}

\pandocbounded{\includegraphics[keepaspectratio]{Exam1-Prep_files/figure-pdf/unnamed-chunk-4-1.pdf}}

\paragraph{Consider the two plots
below:}\label{consider-the-two-plots-below}

\pandocbounded{\includegraphics[keepaspectratio]{Exam1-Prep_files/figure-pdf/unnamed-chunk-5-1.pdf}}

\pandocbounded{\includegraphics[keepaspectratio]{Exam1-Prep_files/figure-pdf/unnamed-chunk-5-2.pdf}}

Compare and contrast how these two visuals explain variability in the
\emph{response} variable. In your answer, mention:

\begin{itemize}
\tightlist
\item
  What the plot shows
\item
  How variation is partitioned or summarized
\item
  What conclusions might be drawn about the role of the explanatory
  variable
\end{itemize}




\end{document}
